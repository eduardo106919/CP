\documentclass[a4paper,11pt]{article}
\usepackage[portuguese]{babel}
\usepackage{graphicx}
\usepackage{amsmath}
\usepackage{minted}
\usepackage{mdframed}
\usepackage{stmaryrd}
\usepackage[all]{xy}
\usepackage{amsfonts}

\usepackage[twoside,verbose,body={16cm,24cm},
left=25mm,top=20mm]{geometry}


\title{Cálculo de Programas \\ Resolução - Ficha 11}
\author{Eduardo Freitas Fernandes}
\date{2025}


\begin{document}
	
	\maketitle
	
	\noindent \underline{\textbf{Exercício 1}}
	\[
	\begin{aligned}
		& mirror = \llparenthesis \, in_2 \cdot \alpha \, \rrparenthesis \\
		\equiv \ &\{\text{Def. $in_2$ , Def. $\alpha$}\}\\
		& mirror = \llparenthesis \, [Leaf, Fork] \cdot (id + swap) \, \rrparenthesis \\
		\equiv \ &\{\text{Absorção-+}\}\\
		& mirror = \llparenthesis \, [Leaf, Fork \cdot swap] \, \rrparenthesis \\
		\equiv \ &\{\text{Universal-cata}\}\\
		& mirror \cdot in_{LTree} = [Leaf, Fork \cdot swap] \cdot (id + (mirror \times mirror)) \\
		\equiv \ &\{\text{Def. $in_{LTree}$, Fusão-+, Absorção-+, Eq-+}\}\\
		&\begin{cases}
			mirror \cdot Leaf = Leaf \\
			mirror \cdot Fork = Fork \cdot swap \cdot (mirror \times mirror)
		\end{cases}\\
		\equiv \ &\{\text{pointwise, Def. composição}\}\\
		&\begin{cases}
			mirror \ (Leaf \ x) = Leaf \ x \\
			mirror \ (Fork \ (l, r)) = Fork \ (swap \ ( mirror \ l, mirror \ r))
		\end{cases}\\
		\equiv \ &\{\text{Def. swap}\}\\
		&\begin{cases}
			mirror \ (Leaf \ x) = Leaf \ x \\
			mirror \ (Fork \ (l, r)) = Fork \ ( mirror \ r, mirror \ l)
		\end{cases}
	\end{aligned}
	\]
	
	\noindent \underline{\textbf{Exercício 2}}
	\[
	\begin{aligned}
		& ........................................................ \\
		\impliedby \ &\{\text{Fusão-cata}\}\\
		& ........................................................ \\
		\equiv \ &\{\text{$\llparenthesis \, g \, \rrparenthesis \cdot in_2 = g \cdot G \ \llparenthesis \, g \, \rrparenthesis$ (F1)}\}\\
		& ........................................................ \\
		\impliedby \ &\{\text{Leibniz}\}\\
		& ........................................................ \\
		\impliedby \ &\{\text{Generalização de $\llparenthesis \, g \, \rrparenthesis$ em f}\}\\
		& ........................................................ \\
	\end{aligned}
	\]
	
	\newpage
	
	\noindent \underline{\textbf{Exercício 3}}\\
	\[
	\begin{aligned}
		& mirror = \llparenthesis \, g \, \rrparenthesis \\
		& mirror = \llparenthesis \, in_2 \cdot \alpha \, \rrparenthesis
	\end{aligned}
	\]
	
	\[
	\begin{aligned}
		& \llparenthesis \, g \, \rrparenthesis = \llparenthesis \, in_2 \cdot \alpha \, \rrparenthesis \\
		\equiv \ &\{\text{Def. $in_2$, Def. $\alpha$}\}\\
		& \llparenthesis \, g \, \rrparenthesis = \llparenthesis \, [Leaf, Fork] \cdot (id + swap) \, \rrparenthesis \\
		\equiv \ &\{\text{Absorção-+}\}\\
		& \llparenthesis \, g \, \rrparenthesis = \llparenthesis \, [Leaf, Fork \cdot swap] \, \rrparenthesis \\
	\end{aligned}
	\]
	
	\noindent Podemos então dizer que $g = [Leaf, Fork \cdot swap]$. Precisamos também de provar também que $id = \llparenthesis \, g \cdot \alpha \, \rrparenthesis$.
	\[
	\begin{aligned}
		& id = \llparenthesis \, g \cdot \alpha \, \rrparenthesis \\
		\equiv \ &\{\text{Def. g, Def. $\alpha$}\}\\
		& id = \llparenthesis \, [Leaf, Fork \cdot swap] \cdot (id + swap) \, \rrparenthesis \\
		\equiv \ &\{\text{Absorção-+}\}\\
		& id = \llparenthesis \, [Leaf, Fork \cdot swap \cdot swap] \, \rrparenthesis \\
		\equiv \ &\{\text{$swap \cdot swap = id$}\}\\
		& id = \llparenthesis \, [Leaf, Fork] \, \rrparenthesis \\
		\equiv \ &\{\text{$[Leaf, Fork] = in_{LTree}$}\}\\
		& id = \llparenthesis \, in_{LTree} \, \rrparenthesis \\
		\equiv \ &\{\text{Reflexão-cata}\}\\
		& True \\
	\end{aligned}
	\]
	
	\noindent Podemos então provar que $mirror$ é o seu próprio isomorfismo:
	\[
	\begin{aligned}
		& \llparenthesis \, g \, \rrparenthesis \cdot \llparenthesis \, in_2 \cdot \alpha \, \rrparenthesis = \llparenthesis \, g \cdot \alpha \, \rrparenthesis \\
		\impliedby \ &\{\text{(F3)}\}\\
		& G \ f \cdot \alpha = \alpha \cdot F \ f \\
		\equiv \ &\{\text{G f = F f = id + f $\times$ f}\}\\
		& (id + f \times f) \cdot (id + swap) = (id + swap) \cdot (id + f \times f) \\
		\equiv \ &\{\text{Functor-+}\}\\
		& (id + (f \times f) \cdot swap) = (id + swap \cdot (f \times f)) \\
	\end{aligned}
	\]
	
	\noindent Através da propriedade grátis da função $swap$ (i.e: $swap \cdot (f \times g) = (g \times f) \cdot swap$), podemos garantir a veracidade desta propriedade.\\
	
	
	\noindent \underline{\textbf{Exercício 4}}
	\[
	\begin{aligned}
		& \llparenthesis \, g \, \rrparenthesis \cdot T \ f = \llparenthesis \, g \cdot B \ (f, id) \, \rrparenthesis \\
		\equiv \ &\{\text{Def-map-cata}\}\\
		& \llparenthesis \, g \, \rrparenthesis \cdot \llparenthesis \, in \cdot B \ (f, id) \, \rrparenthesis = \llparenthesis \, g \cdot B \ (f, id) \, \rrparenthesis \\
		\impliedby \ &\{\text{(F3)}\}\\
		& G \ f \cdot B \ (f, id) = B \ (f, id) \cdot F \ f \\
		\equiv \ &\{\text{?????}\}\\
		& B \ (id, f) \cdot B \ (f, id) = B \ (f, id) \cdot B \ (id, f) \\
		\equiv \ &\{\text{Functor-id-F (para Bi Functores)}\}\\
		& B \ (f, f) = B \ (f, f) \\
	\end{aligned}
	\]
	
	
	\noindent \underline{\textbf{Exercício 5}}
	\[
	\begin{aligned}
		& while \ p \ f \ g = tailr \ ((g + f) \cdot (\neg \cdot p) ?) \\
		\equiv \ &\{\text{Def. tailr}\}\\
		& while \ p \ f \ g = \llbracket join, ((g + f) \cdot (\neg \cdot p) ?) \rrbracket \\
		\equiv \ &\{\text{$\llbracket f, g \rrbracket = f \cdot F \llbracket f, g \rrbracket \cdot g$}\}\\
		& while \ p \ f \ g = join \cdot F \ (while \ p \ f \ g) \cdot ((g + f) \cdot (\neg \cdot p) ?) \\
		\equiv \ &\{\text{Def. join, Def. Functor}\}\\
		& while \ p \ f \ g = [id, id] \cdot (id + (while \ p \ f \ g)) \cdot ((g + f) \cdot (\neg \cdot p) ?) \\
		\equiv \ &\{\text{Absorção-+ (2$\times$)}\}\\
		& while \ p \ f \ g = [g, (while \ p \ f \ g) \cdot f] \cdot (\neg \cdot p) ? \\
		\equiv \ &\{\text{Def. Condicional de McCarthy}\}\\
		& while \ p \ f \ g = (\neg \cdot p) \rightarrow g, (while \ p \ f \ g) \cdot f \\
		\equiv \ &\{\text{pointwise}\}\\
		& while \ p \ f \ g \ x = \textbf{if} \ not \ (p \ x) \ \textbf{then} \ g \ x \ \textbf{else} \ while \ p \ f \ g \ (f \ x)
	\end{aligned}
	\]
	
	
	\noindent \underline{\textbf{Exercício 6}}
	\[
	\begin{aligned}
		& ........................................................ \\
		\equiv \ &\{\text{Def. tailr, Def. hylomorfismo}\}\\
		& ........................................................ \\
		\impliedby \ &\{\text{Leibniz}\}\\
		& ........................................................ \\
		\impliedby \ &\{\text{Fusão-ana, Def. Functor}\}\\
		& ........................................................ 
	\end{aligned}
	\]
	
	\noindent \underline{\textbf{Exercício 7}}\\
	\[
	\begin{aligned}
		& f \bullet [g, h] = [f \bullet g, f \bullet h] \\
		\equiv \ &\{\text{(F9) (3$\times$)}\}\\
		& \mu \cdot T \ f \cdot [g, h] = [\mu \cdot T \ f \cdot g, \mu \cdot T \ f \cdot h] \\
		\equiv \ &\{\text{Fusão-+}\}\\
		& [\mu \cdot T \ f \cdot g, \mu \cdot T \ f \cdot h] = [\mu \cdot T \ f \cdot g, \mu \cdot T \ f \cdot h] \\
	\end{aligned}
	\]
	
	
	
\end{document}