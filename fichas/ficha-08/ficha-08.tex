\documentclass[a4paper,11pt]{article}
\usepackage[portuguese]{babel}
\usepackage{graphicx}
\usepackage{amsmath}
\usepackage{minted}
\usepackage{stmaryrd}
\usepackage[all]{xy}

\usepackage[twoside,verbose,body={16cm,24cm},
left=25mm,top=20mm]{geometry}


\title{Cálculo de Programas \\ Resolução - Ficha 08}
\author{Eduardo Freitas Fernandes}
\date{2025}


\begin{document}
	
	\maketitle

	\noindent \underline{\textbf{Exercício 1}}\\
	
	\begin{minipage}{0.4\textwidth}
		\[
		\begin{aligned}
			& T \ id = id \\
			\equiv \ &\{\text{(F1)}\}\\
			& id \times id = id \\
			\equiv \ &\{\text{Def-$\times$}\}\\
			& \langle id \cdot \pi_1, id \cdot \pi_2 \rangle = id \\
			\equiv \ &\{\text{Natural-id (twice)}\}\\
			& \langle \pi_1, \pi_2 \rangle = id \\
			\equiv \ &\{\text{Reflexão-$\times$}\}\\
			& id = id \\
		\end{aligned}
		\]
	\end{minipage}
	\hfill
	\begin{minipage}{0.6\textwidth}
		\[
		\begin{aligned}
			& T \ (f \cdot g) = T \ f \cdot T \ g \\
			\equiv \ &\{\text{(F1)}\}\\
			& (f \cdot g) \times (f \cdot g) = (f \times f) \cdot (g \times g) \\
			\equiv \ &\{\text{Def-$\times$ (twice)}\}\\
			& \langle f \cdot g \cdot \pi_1, f \cdot g \cdot \pi_2 \rangle = \langle f \cdot \pi_1, f \cdot \pi_2 \rangle \cdot (g \times g) \\
			\equiv \ &\{\text{Fusão-$\times$}\}\\
			& \langle f \cdot g \cdot \pi_1, f \cdot g \cdot \pi_2 \rangle = \langle f \cdot \pi_1 \cdot (g \times g), f \cdot \pi_2 \cdot (g \times g) \rangle \\
			\equiv \ &\{\text{Natural-$\pi_1$, Natural-$\pi_2$}\}\\
			& \langle f \cdot g \cdot \pi_1, f \cdot g \cdot \pi_2 \rangle = \langle f \cdot g \cdot \pi_1, f \cdot g \cdot \pi_2 \rangle
		\end{aligned}
		\]
	\end{minipage}
	
	
	
	\[
	\begin{aligned}
		& \mu \cdot T \ u = \mu \cdot u \\
		\equiv \ &\{\text{Def. $\mu$, Def. T u}\}\\
		& (\pi_1 \times \pi_2) \cdot (\langle id, id \rangle \times \langle id, id \rangle) = (\pi_1 \times \pi_2) \cdot \langle id, id \rangle \\
		\equiv \ &\{\text{Def-$\times$, Absorção-$\times$}\}\\
		& (\pi_1 \times \pi_2) \cdot \langle \langle id, id \rangle \cdot \pi_1, \langle id, id \rangle \cdot \pi_2 \rangle = \langle \pi_1, \pi_2 \rangle \\
		\equiv \ &\{\text{Absorção-$\times$, Reflexão-$\times$}\}\\
		& \langle \pi_1 \cdot \langle id, id \rangle \cdot \pi_1, \pi_2 \cdot \langle id, id \rangle \cdot \pi_2 \rangle = \langle \pi_1, \pi_2 \rangle \\
		\equiv \ &\{\text{Cancelamento-$\times$}\}\\
		& \langle \pi_1, \pi_2 \rangle = \langle \pi_1, \pi_2 \rangle \\
	\end{aligned}
	\]
	
	\newpage
	
	\noindent \underline{\textbf{Exercício 2}}
	\[
	loop \ (a,b) = (2+b, 2-a) \equiv loop = ((2+) \times (2-)) \cdot swap	
	\]
	
	\[
	\begin{aligned}
		& \langle f, g \rangle = for \ loop \ (4,-2) \\
		\equiv \ &\{\text{Def. for}\}\\
		& \langle f, g \rangle = \llparenthesis \, [\underline{(4,-2)}, loop] \, \rrparenthesis \\
		\equiv \ &\{\text{Def. loop}\}\\
		& \langle f, g \rangle = \llparenthesis \, [\underline{4} \times \underline{-2}, (2+) \times (2-) \cdot swap] \, \rrparenthesis \\
		\equiv \ &\{\text{Def-$\times$ (twice), Fusão-$\times$}\}\\
		& \langle f, g \rangle = \llparenthesis \, [\langle \underline{4} \cdot \pi_1, \underline{-2} \cdot \pi_2 \rangle, \langle (2+) \cdot \pi_1 \cdot swap, (2-)\cdot \pi_2 \cdot swap \rangle] \, \rrparenthesis \\
		\equiv \ &\{\text{Lei da Troca}\}\\
		& \langle f, g \rangle = \llparenthesis \, \langle [\underline{4} \cdot \pi_1, (2+) \cdot \pi_1 \cdot swap], [\underline{-2} \cdot \pi_2, (2-) \cdot \pi_2 \cdot swap] \rangle \, \rrparenthesis \\
		\equiv \ &\{\text{Fokkinga}\}\\
		&\begin{cases}
			f \cdot [zero, succ] = [\underline{4} \cdot \pi_1, (2+) \cdot \pi_1 \cdot swap] \cdot (id + \langle f, g \rangle) \\
			g \cdot [zero, succ] = [\underline{-2} \cdot \pi_2, (2-) \cdot \pi_2 \cdot swap] \cdot (id + \langle f, g \rangle)
		\end{cases}\\
		\equiv \ &\{\text{Fusão-+, Absorção-+, Eq-+}\}\\
		&\begin{cases}
			f \cdot zero = \underline{4} \cdot \pi_1 \\
			f \cdot succ = (2+) \cdot \pi_1 \cdot swap \\
			g \cdot zero = \underline{-2} \cdot \pi_2 \\
			g \cdot succ = (2-) \cdot \pi_2 \cdot swap 
		\end{cases}\\
		\equiv \ &\{\text{pointwise}\}\\
		&\begin{cases}
			f \ 0 = 4 \\
			f \ (n+1) = 2 + g \ n \\
			g \ 0 = -2 \\
			g \ (n+1) = 2 - f \ n
		\end{cases}
	\end{aligned}
	\]
	
	
	\noindent \underline{\textbf{Exercício 3}}
	\[
	\begin{aligned}
		& \langle \langle f, g \rangle, j \rangle = \llparenthesis \, \langle \langle h, k \rangle, l \rangle \, \rrparenthesis \\
		\equiv \ &\{\text{Fokkinga}\}\\
		&\begin{cases}
			\langle f, g \rangle \cdot in = \langle h, k \rangle \cdot F \ \langle \langle f, g \rangle, j \rangle \\
			j \cdot in = l \cdot F \ \langle \langle f, g \rangle, j \rangle
		\end{cases}\\
		\equiv \ &\{\text{Fusão-$\times$ (twice)}\}\\
		&\begin{cases}
			\langle f \cdot in, g \cdot in \rangle = \langle h \cdot F \ \langle \langle f, g \rangle, j \rangle, k \cdot F \ \langle \langle f, g \rangle, j \rangle \rangle \\
			j \cdot in = l \cdot F \ \langle \langle f, g \rangle, j \rangle
		\end{cases}\\
		\equiv \ &\{\text{Eq-$\times$}\}\\
		&\begin{cases}
			f \cdot in = h \cdot F \ \langle \langle f, g \rangle, j \rangle \\
			g \cdot in = k \cdot F \ \langle \langle f, g \rangle, j \rangle \\
			j \cdot in = l \cdot F \ \langle \langle f, g \rangle, j \rangle
		\end{cases}\\
	\end{aligned}
	\]
	
	
	\noindent \underline{\textbf{Exercício 4}}
		\[
		\begin{aligned}
			&\begin{cases}
				impar \ 0 = false \\
				impar \ (n+1) = par \ n \\
				par \ 0 = true \\
				par \ (n+1) = impar \ n
			\end{cases}\\
			\equiv \ &\{\text{pointfree}\}\\
			&\begin{cases}
				impar \cdot zero = False \\
				impar \cdot succ = par \\
				par \cdot zero = True \\
				par \cdot succ = impar
			\end{cases}\\
			\equiv \ &\{\text{Eq-+}\}\\
			&\begin{cases}
				[impar \cdot zero, impar \cdot succ] = [False, par] \\
				[par \cdot zero, par \cdot succ] = [True, impar]
			\end{cases}\\
			\equiv \ &\{\text{Fusão-+, Cancelamento-+ (twice)}\}\\
			&\begin{cases}
				impar \cdot in = [False, \pi_2 \cdot \langle impar, par \rangle] \\
				par \cdot in = [True, \pi_1 \cdot \langle impar, par \rangle]
			\end{cases}\\
			\equiv \ &\{\text{Natural-id (twice), Absorção-+}\}\\
			&\begin{cases}
				impar \cdot in = [False, \pi_2] \cdot (id + \langle impar, par \rangle) \\
				par \cdot in = [True, \pi_1] \cdot (id + \langle impar, par \rangle) \\
			\end{cases}\\
		\end{aligned}
		\]

	\noindent Podemos então concluir que $h = [False, \pi_2]$ e $k = [True, \pi_1]$.\\

		\[
		\begin{aligned}
			&\langle impar, par \rangle = for \ swap \ (False, True) \\
			\equiv \ &\{\text{Def. for}\}\\
			&\langle impar, par \rangle = \llparenthesis \, [False \times True, swap] \rrparenthesis \\
			\equiv \ &\{\text{Def. swap, Def-$\times$}\}\\
			&\langle impar, par \rangle = \llparenthesis \, [\langle False \cdot \pi_1, True \cdot \pi_2 \rangle, \langle \pi_2, \pi_1 \rangle] \, \rrparenthesis \\
			\equiv \ &\{\text{Lei da Troca}\}\\
			&\langle impar, par \rangle = \llparenthesis \, \langle [False \cdot \pi_1, \pi_2], [True \cdot \pi_2, \pi_1] \rangle \, \rrparenthesis \\
			\equiv \ &\{\text{Fokkinga}\}\\
			&\begin{cases}
				impar \cdot [zero, succ] = [False \pi_1, \pi_2] \cdot (id + \langle impar, par \rangle) \\
				par \cdot [zero, succ] = [True \cdot \pi_2, \pi_1] \cdot (id + \langle impar, par \rangle)
			\end{cases}\\
			\equiv \ &\{\text{Fusão-+, Absorção-+}\}\\
			&\begin{cases}
				[impar \cdot zero, impar \cdot succ] = [False \cdot \pi_1, \pi_2 \cdot \langle impar, par \rangle] \\
				[par \cdot zero, par \cdot succ] = [True \cdot \pi_2, \pi_1 \cdot \langle impar, par \rangle]
			\end{cases}\\
			\equiv \ &\{\text{Cancelamento-$\times$ (twice), Eq-+, pointwise}\}\\
			&\begin{cases}
				impar \ 0 = False \\
				impar \ (n+1) = par \ n \\
				par \ 0 = True \\
				par \ (n+1) = impar \ n
			\end{cases}
		\end{aligned}
		\]
	
	
	
	
	\noindent \underline{\textbf{Exercício 5}}
		\[
		\begin{aligned}
			&\begin{cases}
				insg \ 0 = \ [ \ ] \\
				insg \ (n+1) = fsuc + 1 : insg \ n \\
				fsuc \ 0 = 1 \\
				fsuc \ (n+1) = fsuc \ n + 1
			\end{cases} \\
			\equiv \ &\{\text{pointwise}\}\\
			&\begin{cases}
				insg \cdot zero = nil \\
				insg \cdot succ = cons \cdot \langle fsuc, insg \rangle \\
				fsuc \cdot zero = one \\
				fsuc \cdot succ = (1+) \cdot fsuc
			\end{cases} \\
			\equiv \ &\{\text{Cancelamento-$\times$}\}\\
			&\begin{cases}
				insg \cdot zero = nil \\
				insg \cdot succ = cons \cdot \langle fsuc, insg \rangle \\
				fsuc \cdot zero = one \\
				fsuc \cdot succ = (1+) \cdot \pi_1 \cdot \langle fsuc, insg \rangle
			\end{cases} \\
			\equiv \ &\{\text{Eq-+, Fusão-+, Absorção-+}\}\\
			&\begin{cases}
				insg \cdot in = [nil, cons] \cdot (id + \langle fsuc, insg \rangle) \\
				fsuc \cdot in = [one, (1+) \cdot \pi_1] \cdot (id + \langle fsuc, insg \rangle)
			\end{cases} \\
			\equiv \ &\{\text{Fokkinga}\}\\
			& \langle fsuc, insg \rangle = \llparenthesis \, \langle in_*, [one, (1+) \cdot \pi_1] \rangle \rrparenthesis
		\end{aligned}
		\]
	
		\[
		\begin{aligned}
			& insgfor = for \ \langle (1+) \cdot \pi_1, cons \rangle \ \langle one, nil \rangle \\
			\equiv \ &\{\text{Def. for}\}\\
			& insgfor = \llparenthesis \, [\langle one, nil \rangle, \langle (1+) \cdot \pi_1, cons \rangle] \, \rrparenthesis \\
			\equiv \ &\{\text{Lei da Troca}\}\\
			& insgfor = \llparenthesis \, \langle [one, (1+) \cdot \pi_1], [nil, cons] \rangle \, \rrparenthesis \\
			\equiv \ &\{\text{$insg = \pi_2 \cdot insgfor \implies insgfor = \langle f, insg \rangle$}\}\\
			& \langle f, insg \rangle = \llparenthesis \, \langle [one, (1+) \cdot \pi_1], [nil, cons] \rangle \, \rrparenthesis \\
			\equiv \ &\{\text{Fokkinga}\}\\
			&\begin{cases}
				f \cdot [zero, succ] = [one, (1+) \cdot \pi_1] \cdot F \ \langle f, insg \rangle \\
				insg \cdot [zero, succ] = [nil, cons] \cdot F \ \langle f, insg \rangle \\
			\end{cases}\\
			\equiv \ &\{\text{Functor dos Naturais, Fusão-+, Absorção-+, Eq-+}\}\\
			&\begin{cases}
				f \cdot zero = one \\
				f \cdot succ = (1+) \cdot \pi_1 \cdot \langle f, insg \rangle \\
				insg \cdot zero = nil \\
				insg \cdot succ = cons \cdot \langle f, insg \rangle \\
			\end{cases}\\
			\equiv \ &\{\text{pointwise}\}\\
			&\begin{cases}
				f \ 0 = 1 \\
				f \ (n+1) = 1 + f \ n \\
				insg \ 0 = [ \ ] \\
				insg \ (n+1) = f \ n \ : \ insg \ n
			\end{cases}
		\end{aligned}
		\]
		
	
	\noindent \underline{\textbf{Exercício 6}}
	\[
	\begin{aligned}
		&\begin{cases}
			f_1 \ [ \ ] = [ \ ] \\
			f_1 \ (h:t) = h \ : \ f_2 \ t \\
			f_2 \ [ \ ] = [ \ ] \\
			f_1 \ (h:t) = f_1 \ t \\
		\end{cases}\\
		\equiv \ &\{\text{Def. composição, pointfree}\}\\
		&\begin{cases}
			f_1 \cdot nil = nil \\
			f_1 \cdot cons = cons \cdot (id \times f_2) \\
			f_2 \cdot nil = nil \\
			f_2 \cdot cons = f_1 \cdot \pi_2
		\end{cases}\\
		\equiv \ &\{\text{Eq-+, Fusão-+}\}\\
		&\begin{cases}
			f_1 \cdot [nil, cons] = [nil, cons \cdot (id \times f_2)] \\
			f_2 \cdot [nil, cons] = [nil, f_1 \cdot \pi_2]
		\end{cases}\\
		\equiv \ &\{\text{Def. in, Natural-id (2*), Natural-$\pi_2$, Cancelamento-$\times$}\}\\
		&\begin{cases}
			f_1 \cdot in = [nil \cdot id, cons \cdot (id \times \pi_2 \cdot \langle f_1, f_2 \rangle )] \\
			f_2 \cdot in = [nil \cdot id, \pi_2 \cdot (id \times \pi_1 \cdot \langle f_1, f_2 \rangle)]
		\end{cases}\\
		\equiv \ &\{\text{Functor-$\times$}\}\\
		&\begin{cases}
			f_1 \cdot in = [nil \cdot id, cons \cdot (id \times \pi_2) \cdot (id \times \langle f_1, f_2 \rangle)] \\
			f_2 \cdot in = [nil \cdot id, \pi_2 \cdot (id \times \pi_1) \cdot (id \times \langle f_1, f_2 \rangle)]
		\end{cases}\\
		\equiv \ &\{\text{Absorção-+, Functor de listas}\}\\
		&\begin{cases}
			f_1 \cdot in = [nil, cons \cdot (id \times \pi_2)] \cdot F \ \langle f_1, f_2 \rangle \\
			f_2 \cdot in = [nil, \pi_2 \cdot (id \times \pi_1)] \cdot F \ \langle f_1, f_2 \rangle \\
		\end{cases}\\
		\equiv \ &\{\text{Fokkinga}\}\\
		& \langle f_1, f_2 \rangle = \llparenthesis \, \langle in_* \cdot (id + (id \times \pi_2)), [nil, \pi_2 \cdot (id \times \pi_1)] \rangle \, \rrparenthesis \\
		\equiv \ &\{\text{Natural-$\pi_2$}\}\\
		& \langle f_1, f_2 \rangle = \llparenthesis \, \langle in_* \cdot (id + (id \times \pi_2)), [nil, \pi_1 \cdot \pi_2] \rangle \, \rrparenthesis
	\end{aligned}
	\]
	
	\[
	\xymatrix@C=2cm{
		A^*
		\ar[d]_-{\langle f_1, f_2 \rangle}
		\ar@/^1pc/[r]^-{out}
		&
		a + A \times A^*
		\ar[d]^{id + id \times \langle f_1, f_2 \rangle}
		\ar@/^1pc/[l]^-{in}
		\\
		A^* \times A^*
		&
		1 + A \times (A^* \times A^*)
		\ar[l]^-{g}
	}
	\]
	
	\noindent Diagrama do gene do catamorfismo:
	
	\[
	\xymatrix{
		& 1+ A \times (A^* \times A^*) \ar[dl]_{id + (id \times \pi_2)} \ar[dr]^{id + \pi_2} & \\
		1 + A \times A^* \ar[d]^{in_*} & & 1 + (A^* \times A^*) \ar[d]_{[nil, \pi_1]} \\
		A^* & & A^*
	}
	\]
	
	\noindent A função $f_1$ seleciona os elementos de uma lista nas posições pares, e a função $f_2$ seleciona os elementos de uma lista nas posições ímpares.
	
	\noindent \underline{\textbf{Exercício 7}}
	
	\begin{minipage}{0.5\textwidth}
		\[
		\begin{aligned}
			& H \ (g \cdot h) = (H \ g) \cdot (H \ h) \\
			\equiv \ &\{\text{Functor-H (3*)}\}\\
			& F \ (g \cdot h) + G \ (g \cdot h) = (F \ g + G \ g) \cdot (F \ h + G \ h) \\
			\equiv \ &\{\text{Functor-F (3*), Functor-G (3*)}\}\\
			& id + (g \cdot h) = (id + g) \cdot (id + h) \\
			\equiv \ &\{\text{Natural-id, Functor-+}\}\\
			&(id \cdot id) + (g \cdot h) = (id \cdot id) + (g \cdot h)
		\end{aligned}
		\]
	\end{minipage}
	\hfill
	\begin{minipage}{0.5\textwidth}
		\[
		\begin{aligned}
			& H \ id = id \\
			\equiv \ &\{\text{Functor-H}\}\\
			& F \ id + G \ id = id \\
			\equiv \ &\{\text{Functor-F, Functor-G}\}\\
			& id + id = id \\
			\equiv \ &\{\text{Functor-id-+}\}\\
			& id = id \\
		\end{aligned}
		\]
	\end{minipage}
	
	\begin{minipage}{0.5\textwidth}
		\[
		\begin{aligned}
			& K \ (g \cdot h) = (K \ g) \cdot (K \ h) \\
			\equiv \ &\{\text{Functor-K}\}\\
			& G \ (g \cdot h) \times F \ (g \cdot h) = (G \ g \times F \ g) \cdot (G \ h \times F \ h) \\
			\equiv \ &\{\text{Functor-F (3*), Functor-G (3*)}\}\\
			& (g \cdot h) \times id = (g \times id) \cdot (h \times id) \\
			\equiv \ &\{\text{Natural-id, Functor-$\times$}\}\\
			& (g \cdot h) \times (id \cdot id) = (g \cdot h) \times (id \cdot id) \\
		\end{aligned}
		\]
	\end{minipage}
	\hfill
	\begin{minipage}{0.5\textwidth}
		\[
		\begin{aligned}
			& K \ id = id \\
			\equiv \ &\{\text{Functor-K}\}\\
			& G \ id \times F \ id = id \\
			\equiv \ &\{\text{Functor-F, Functor-G}\}\\
			& id \times id = id \\
			\equiv \ &\{\text{Functor-id-$\times$}\}\\
			& id = id \\
		\end{aligned}
		\]
	\end{minipage}
	\\
	
	\noindent \underline{\textbf{Exercício 8}}
	
	\begin{minipage}{0.4\textwidth}
		\[
		\begin{aligned}
			& H \ id = id \\
			\equiv \ &\{\text{Definição do Functor H}\}\\
			& (F \cdot G) \ id = id \\
			\equiv \ &\{\text{Composição de Functores}\}\\
			& F \ (G \ id) = id \\
			\equiv \ &\{\text{Functor-id-G}\}\\
			& F \ id = id \\
			\equiv \ &\{\text{Functor-id-F}\}\\
			& id = id \\
		\end{aligned}
		\]
	\end{minipage}
	\hfill
	\begin{minipage}{0.5\textwidth}
		\[
		\begin{aligned}
			& H \ (f \cdot g) = (H \ f) \cdot (H \ g)  \\
			\equiv \ &\{\text{Definição do Functor H}\}\\
			& (F \cdot G) \ (f \cdot g) = ((F \cdot G) \ f) \cdot ((F \cdot G) \ g) \\
			\equiv \ &\{\text{Composição de Functores}\}\\
			& F \ (G \ (f \cdot g)) = (F \ (G \ f)) \cdot (F \ (G \ g)) \\
			\equiv \ &\{\text{Functor-G, Functor-F}\}\\
			& F \ (G \ f \cdot G \cdot g) = F \ (G \ f \cdot G \cdot g) \\
		\end{aligned}
		\]
	\end{minipage}
	
	\noindent \underline{\textbf{Exercício 9}}
	\[
	\begin{aligned}
		& unzip \ xs = (map \ \pi_1 \ xs, map \ \pi_2 \ xs) \\
		\equiv \ &\{\text{pointfree}\}\\
		& unzip = \langle map \ \pi_1, map \ \pi_2 \rangle \\
		\equiv \ &\{\text{Def-map-cata}\}\\
		& unzip = \langle \llparenthesis \, in \cdot B \ (\pi_1, id) \, \rrparenthesis, \llparenthesis \, in \cdot B \ (\pi_2, id) \, \rrparenthesis \rangle \\
		\equiv \ &\{\text{Banana-split}\}\\
		& unzip = \llparenthesis \, ((in \cdot B \ (\pi_1, id)) \times (in \cdot B \ (\pi_2, id))) \cdot \langle F \ \pi_1, F \ \pi_2 \rangle \, \rrparenthesis \\
		\equiv \ &\{\text{Absorção-$\times$}\}\\
		& unzip = \llparenthesis \, \langle in \cdot B \ (\pi_1, id) \cdot F \ \pi_1, in \cdot B \ (\pi_2, id) \cdot F \ \pi_2 \rangle \, \rrparenthesis \\
		\equiv \ &\{\text{Base-cata (twice), Functor-B (para Bi Functores)}\}\\
		& unzip = \llparenthesis \, \langle in \cdot B \ (\pi_1, \pi_1), in \cdot B \ (\pi_2, \pi_2) \rangle \, \rrparenthesis \\
		\equiv \ &\{\text{Universal-cata}\}\\
		& unzip \cdot in = \langle in \cdot B \ (\pi_1, \pi_1), in \cdot B \ (\pi_2, \pi_2) \rangle \cdot F \ unzip \\
		\equiv \ &\{\text{Def. $in_*$, Def. Bi-Functor de listas}\}\\
		& unzip \cdot in = \langle [nil, cons] \cdot (id + \pi_1 \times \pi_1), [nil, cons] \cdot (id + \pi_2 \times \pi_2) \rangle \cdot F \ unzip \\
		\equiv \ &\{\text{Absorção-+, Definição do Functor de Listas}\}\\
		& unzip \cdot in = \langle [nil, cons \cdot (\pi_1 \times \pi_1)], [nil, cons \cdot (\pi_2 \times \pi_2)] \rangle \cdot (id + id \times unzip) \\
		\equiv \ &\{\text{Lei da Troca}\}\\
		& unzip \cdot in = [ \langle nil, nil \rangle , \langle cons \cdot (\pi_1 \times \pi_1), cons \cdot (\pi_2 \times \pi_2) \rangle ] \cdot (id + id \times unzip) \\
		\equiv \ &\{\text{Absorção-+, Natural-id}\}\\
		& unzip \cdot in = [ \langle nil, nil \rangle , \langle cons \cdot (\pi_1 \times \pi_1), cons \cdot (\pi_2 \times \pi_2) \rangle \cdot (id \times unzip) ] \\
		\equiv \ &\{\text{Def. in, Fusão-+, Eq-+}\}\\
		& \begin{cases}
			unzip \cdot nil = \langle nil, nil \rangle \\
			unzip \cdot cons = \langle cons \cdot (\pi_1 \times \pi_1), cons \cdot (\pi_2 \times \pi_2) \rangle \cdot (id \times unzip)
		\end{cases}\\
		\equiv \ &\{\text{Absorção-$\times$}\}\\
		& \begin{cases}
			unzip \cdot nil = \langle nil, nil \rangle \\
			unzip \cdot cons = (cons \times cons) \cdot \langle \pi_1 \times \pi_1, \pi_2 \times \pi_2 \rangle \cdot (id \times unzip)
		\end{cases}\\
		\equiv \ &\{\text{pointwise}\}\\
		&\begin{cases}
			unzip \ [ \ ] = ([ \ ], [ \ ]) \\
			unzip \ ((a,b):xs) = (cons \times cons) \cdot \langle \pi_1 \times \pi_1, \pi_2 \times \pi_2 \rangle \cdot (id \times unzip) \ ((a,b):xs)
		\end{cases}\\
		\equiv \ &\{\text{Definição do produto de funções, Definição de id}\}\\
		&\begin{cases}
			unzip \ [ \ ] = ([ \ ], [ \ ]) \\
			unzip \ ((a,b):xs) = (cons \times cons) \cdot \langle \pi_1 \times \pi_1, \pi_2 \times \pi_2 \rangle \ ((a,b), (as, bs)) \ \textbf{where} \ (as,bs) = unzip \ xs
		\end{cases}\\
		\equiv \ &\{\text{Definição de split, De. $\pi_1$ e $\pi_2$}\}\\
		&\begin{cases}
			unzip \ [ \ ] = ([ \ ], [ \ ]) \\
			unzip \ ((a,b):xs) = (cons \times cons) \ ( (a, as) , (b,bs) ) \ \textbf{where} \ (as,bs) = unzip \ xs
		\end{cases}\\
		\equiv \ &\{\text{Definição do produto de funções, Def. cons}\}\\
		&\begin{cases}
			unzip \ [ \ ] = ([ \ ], [ \ ]) \\
			unzip \ ((a,b):xs) = ((a:as), (b:bs)) \ \textbf{where} \ (as,bs) = unzip \ xs
		\end{cases}\\
	\end{aligned}
	\]
	
		
\end{document}